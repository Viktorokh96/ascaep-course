\chapter{РЕАЛИЗАЦИЯ}

\section{Программное обеспечение КУСД}

Работа КУСД, как было сказано ранее (стр. \pageref{reasonofwork}), полностью зависит от поддержки стека сетевых протоколов TCP/IP. Следовательно, основная часть данной работы направлена на создание программного обеспечения, реализующего поддержку стека сетевых протоколов.

В качестве инструмента реализации программного обеспечения использовался домашний компьютер с ОС Linux Ubuntu 16.04, на котором было установлен следующее ПО:
 \begin{itemize}
	\item набор GNU AVR Toolchain версии 4.9.2, в который входит\cite{avrtoolchain}:
	\begin{itemize}
		\item[•] компилятор avr-gcc, для комплиляции исходного кода на языке С/С++ в машинный язык микроконтроллеров семейства AVR;
		\item[•] набор ассемблера и компоновщика avr-binutils;
		\item[•] avr-libc --- подмножество стандартной библиотеки С с некоторыми специфичными для AVR функциями;
		\item[•] avr-gdb --- отладчик;
		\item[•] avrdude --- программа для загрузки/выгрузки/управления памятью программ и данных на МК AVR;
	\end{itemize}
	\item интегрированная среда разработки Elcipse версии 3.8.1 с плагином для разработки программ на МК AVR;
\end{itemize}

\section{Работа с Ethernet-контроллером}

Первоначально требовалось обеспечить передачу данных между Ethernet-контроллером и микроконтроллером ATmega328/P по шине SPI. Опираясь на таблицу \ref{spitabl}, а также на расположение выводов контроллера enc28j60 (рисунок \ref{fig:ethcontroller}), микроконтроллер был подключен к Ethernet-контроллеру. Соответсвие выходам микроконтроллера ATmega328/P и входам Ethernet-контроллера описано в таблице \ref{connection}. Стоит отметить что хоть Ethernet-контроллер в основном оперирует напряжением 3,3 В, на вход питания можно подавать 5 В и выше, поскольку на плате присутствует стабилизатор напряжения AMS1117\cite{voltageregulator}. После подключение убеждаемся в том, что на плату Ethernet-контроллера поступает питание по светодиду D1. 

\begin{table}[h!]
\caption{Соответсвие выходам ATmega328/P и входам Ethernet-контроллера в данной работе}
\label{connection}
	\begin{tabular}{|p{40mm}|p{40mm}|p{40mm}|}
\hline
	Наименование соединения & Выход на ATmega328/P & Вход на Ethernet-контроллере \\
\hline
		MOSI & PB3 & MOSI (4)\\
\hline
		MISO & PB4 & MISO(10)\\
\hline
		SCLK & PB5 & SCK(9)\\
\hline
		SS & PB2 & CS (5)\\
\hline
\end{tabular}
\end{table}

%Что бы в дальнейшем осуществлять передачу данных между МК и Ethernet-контроллером необходимо выяснить устройство контроллера enc28j60 подробнее.

\subsection{Архитектура ENC28J60}


