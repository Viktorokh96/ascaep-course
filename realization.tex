\chapter{РЕАЛИЗАЦИЯ}

\section{Программное обеспечение КУСД}

Работа КУСД, как было сказано ранее (стр. \pageref{reasonofwork}), полностью зависит от поддержки стека сетевых протоколов TCP/IP. Следовательно, основная часть данной работы направлена на создание программного обеспечения, реализующего поддержку стека сетевых протоколов.

В качестве инструмента реализации программного обеспечения использовался домашний компьютер с ОС Linux Ubuntu 16.04, на котором было установлен следующее ПО:
 \begin{itemize}
	\item набор GNU AVR Toolchain версии 4.9.2, в который входит\cite{avrtoolchain}:
	\begin{itemize}
		\item[•] компилятор avr-gcc, для комплиляции исходного кода на языке С/С++ в машинный язык микроконтроллеров семейства AVR;
		\item[•] набор ассемблера и компоновщика avr-binutils;
		\item[•] avr-libc --- подмножество стандартной библиотеки С с некоторыми специфичными для AVR функциями;
		\item[•] avr-gdb --- отладчик;
		\item[•] avrdude --- программа для загрузки/выгрузки/управления памятью программ и данных на МК AVR;
	\end{itemize}
	\item интегрированная среда разработки Elcipse версии 3.8.1 с плагином для разработки программ на МК AVR;
\end{itemize}
